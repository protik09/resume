\cvsection{Academic Projects}
\begin{cventries}
	\cventry
	{Dual Degree Project, IIT Madras}
	{Asynchronous Wireless Communication System}
	{Chennai, India}
	{Aug. 2014 - May. 2015}
	{
		\begin{cvitems}
		\item{Worked with \href{http://www.ee.iitm.ac.in/~rvenkat/}{Prof. Venkatesh Ramaiyan} to develop asynchronous short packet communication system.}
		\item{Implemented half duplex short packet communication system on USRP N210 systems, from scratch, in an extensible manner.}
		\item{Used quaternary Barker sequence and Root Raised Cosine pulses to minimize inter-symbol interference in transmitter design.}
		\item{Used Harris Rice synchronization algorithm for optimal timing acquisition, maximum likelihood method for carrier frequency offset estimation, blockwise correlation to find the Barker sequence, in receiver design.}
		\item{Analysed receiver performance against known SNR values, and known levels of asynchronism in packet transmission.}
		\item{Confirmed theoretical results such as exponential dependence of asynchronism threshold on packet length for uniform arrival process.}
		\end{cvitems}
	}

	\cventry
	{Course Project (\href{http://www.ee.iitm.ac.in/2015/03/modern-coding-theory-ee5161/}{Modern Coding Theory}), IIT Madras}
	{LDPC Decoder Analysis}
	{Chennai, India}
	{Aug. 2013 - Nov. 2013}
	{
		\begin{cvitems}
		\item{Developed Low Density Parity Check (LDPC) decoders using Tanner Graph of 15,000 nodes, built with modified Progressive Edge Growth algorithm.}
		\item{Designed decoders and analysed performance across Binary Erasure Channels and Binary Symmetric Channels, for ranges of bit flip/loss probabilities.}
		\item{Designed decoder and analysed performance across BPSK over AWGN Channels, for ranges of SNR values.}
		\item{Verified asymptotic approach of coding scheme performance to Shannon theoretic limit.}
		\end{cvitems}
	}

	\cventry
	{Course Project (Digital IC Design), IIT Madras}
	{Cell Optimization}
	{Chennai, India}
	{Aug. 2013 - Nov. 2013}
	{
		\begin{cvitems}
		\item{Gained hands-on experience designing various gates in Magic.}
		\item{Designed and optimized in 180 nm, a D-flip-flop, a 2-1 multiplexer, and an or-and-invert gate, with the tools Magic and SpiceOpus, using a configuration file from TSMC.}
		\end{cvitems}
	}

\end{cventries}
